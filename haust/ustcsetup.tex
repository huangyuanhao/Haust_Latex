% !TeX root = ./main.tex

% 注意:\ustcsetup 的键值列表中,最后一项后面不要留“逗号”,否则在部分 LaTeX 发行版/编辑器
% 组合下会触发 keyval 解析错误(例如:Paragraph ended before \kv@processor@default ...)。
\ustcsetup{
  title           = {河南科技大学latex模板},
  title*          = {Henan University of Science and Technology LaTeX Template},
  author          = {姓名},
  author*         = {},
  % ===== 河科大封面字段(可按需填写) =====
  clc             = {\hspace{6em}},
  udc             = {\hspace{6em}},
  serial-number   = {\hspace{6em}},
  degree-category  = {工学博士},
  degree-category* = {Master of Engineering},
  % 专业学位填写(可留空)
  co-supervisor    = {},
  professional-field  = {},
  professional-field* = {},
  speciality      = {控制科学与工程},
  sub-speciality  = {系统工程},
  supervisor      = {XXX~教授},
  % ===== 河科大摘要字段(可按需填写) =====
%  thesis-type     = {应用研究},
%  thesis-type*    = {Applied research},
%  subject-source  = {国家自然科学基金项目(示例)},
%  subject-source* = {NSFC project (example)},
  % 数学字体
  % math-style      = GB,  % 可选:GB, TeX, ISO
  math-font       = xits  % 可选:stix, xits, libertinus
}


% ===== 封面成文时间(手动填写到“月”)=====
% 示例:\coverdate{2024 年 12 月}
\coverdate{2026 年 1 月}


% 加载宏包

% 定理类环境宏包
\usepackage{amsthm}

% 插图
\usepackage{graphicx}

% 英文图题、表题
\usepackage{bicaption}

% 三线表
\usepackage{booktabs}

% 表注
\usepackage{threeparttable}

% 跨页表格
\usepackage{longtable}

% 算法
\usepackage[ruled,linesnumbered]{algorithm2e}

% SI 量和单位
\usepackage{siunitx}

% 参考文献使用 BibTeX + natbib 宏包
% 顺序编码制
\usepackage[sort]{natbib}
\bibliographystyle{ustcthesis-numerical}

% 著者-出版年制
% \usepackage{natbib}
% \bibliographystyle{ustcthesis-authoryear}

% 本科生参考文献的著录格式
% \usepackage[sort]{natbib}
% \bibliographystyle{ustcthesis-bachelor}

% 参考文献使用 BibLaTeX 宏包
% \usepackage[style=ustcthesis-numeric]{biblatex}
% \usepackage[bibstyle=ustcthesis-numeric,citestyle=ustcthesis-inline]{biblatex}
% \usepackage[style=ustcthesis-authoryear]{biblatex}
% \usepackage[style=ustcthesis-bachelor]{biblatex}
% 声明 BibLaTeX 的数据库
% \addbibresource{bib/ustc.bib}

% 配置图片的默认目录
\graphicspath{{figures/}}

% 数学命令
\makeatletter
\providecommand\dif{%  % 微分符号
  \mathop{}\!%
  \ifustc@math@style@TeX
    d%
  \else
    \mathrm{d}%
  \fi
}
\makeatother
\providecommand\eu{{\symup{e}}}
\providecommand\iu{{\symup{i}}}

% 用于写文档的命令
\DeclareRobustCommand\cs[1]{\texttt{\char`\\#1}}
\DeclareRobustCommand\env[1]{\texttt{#1}}
\DeclareRobustCommand\pkg[1]{\textsf{#1}}
\DeclareRobustCommand\file[1]{\nolinkurl{#1}}

% hyperref 宏包在最后调用
\usepackage{hyperref}


